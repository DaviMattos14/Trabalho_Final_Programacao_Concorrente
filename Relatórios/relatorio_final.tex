\documentclass{article}
\usepackage[utf8]{inputenc}
\usepackage{graphicx}

\title{Leitura de arquivo para contagem de palavras}
\author{Davi dos Santos Mattos \\
        Daniel Li Vam Man \\
        Pedro André Alves Chaves}
\date{Relatório Parcial \\ Programação Concorrente (ICP-361) - 2025/2}

\begin{document}

\maketitle

\section*{1. Descrição do problema geral}

O problema escolhido é a contagem de palavras em um \textit{arquivo.txt} . Uma tafera muito comum em aplicações de análise de dados, mineração de texto, recuperação da informação e processamento de linguagem neural.

Uma palavra consiste numa sequência de caracteres do alfabeto e a contagem ignora se a letra está em maiúsculo ou minúsculo. As pontuações e espaços indicam o começo e fim de uma palavra.
\begin{itemize}
    \item \textbf{Entrada}: Arquivo de Texto (.txt) qualquer.
    \item \textbf{Saída}: Número de palavras encontradas no arquivo
\end{itemize}

O problema se beneficia da concorrência quando temos que lidar com arquivos de texto muito grandes, pois podemos dividir o arquivo em partes e cada thread executará sua tarefa de contagem de forma paralela, reduzindo dessa forma o tempo de processamento.
    


\section*{2. Projeto e implementação da solução concorrente}

Descrever o projeto da solução concorrente para o problema:

\begin{itemize}
    \item apontar ajustes feitos no projeto inicial (se for o caso);
    \item descrever as principais decisões de implementação adotadas e suas justificativas.
\end{itemize}

Não mostrar o código-fonte, ele será visto diretamente. O objetivo aqui é descrever
e justificar as decisões de projeto e implementação tomadas.

\section*{3. Testes de corretude}

Descrever como o programa foi testado:

\begin{itemize}
    \item descrever o conjunto de casos de teste usados para avaliação da corretude da
    solução proposta (lembrar de variar a dimensão dos dados de entrada e o número
    de threads usadas) e os resultados obtidos.
\end{itemize}

Não mostrar as telas de execução, o objetivo aqui é descrever como os testes foram
feitos e os resultados obtidos.

\section*{4.Avaliação de desempenho}

Descrever como o ganho de desempenho foi avaliado:

\begin{itemize}
    \item descrever o conjunto de casos de teste usados para a avaliação de desempenho
    (lembrar de variar a dimensão dos dados de entrada e o número de threads usadas);
    \item descrever a configuração da máquina onde os testes foram realizados (identificação
    do processador, quantidade de núcleos de execução, sistema operacional);
    \item descrever quantas vezes o mesmo caso de teste foi executado e qual medida de
    tempo foi escolhida;
    \item mostrar o cálculo da aceleração e eficiência e/ou outras métricas pertinentes ao
    problema;
    \item apresentar os resultados obtidos condensados na forma de tabelas ou gráficos.
\end{itemize}

Não mostrar as telas de execução, o objetivo aqui é apresentar os resultados finais
obtidos (tempo de execução, aceleração, eficiência, etc.).

\section*{5. Discussão}

Apresentar uma análise dos resultados obtidos:

\begin{itemize}
    \item discutir se o ganho de desempenho — ou as vantagens pretendidas com a solução
    concorrente — foram alcançadas ou não e por quais motivos;
    \item apresentar possíveis melhorias do programa, se for o caso;
    \item discutir outras questões que forem pertinentes;
    \item descrever dificuldades encontradas para a realização do trabalho, se for o caso.
\end{itemize}

\section*{6. Referências bibliográficas}
\begin{thebibliography}{9}
    \bibitem{exemplo1}
    Rossetto, S. \textbf{Slides de Aula}. 

    \bibitem{exemplo2}
    Maratona de programação paralela (Mackenzie). Disponível em: http://lspd.mackenzie.br/marathon/old.html. 

    \bibitem{exemplo3}
    P. Pacheco, \textbf{An Introduction to Parallel Programming}, Morgan Kaufmann, 2011.

    \bibitem{exemplo4}
    Martin Porter’s Stemming algorithm as a C library. Disponível em: https://github.com/wooorm/stmr.c

    
\end{thebibliography}

\end{document}